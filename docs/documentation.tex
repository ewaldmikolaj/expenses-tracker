\documentclass{article}
\usepackage[utf8]{inputenc}
\usepackage[T1]{fontenc}
\usepackage[default]{opensans}
\usepackage{geometry}

\geometry{
 a4paper,
 total={170mm,257mm},
 left=20mm,
 top=20mm,
 }

%\renewcommand*\familydefault{\sfdefault}

\title{%
    \textbf{expenses-tracker} \\
    \large Projekt zaliczeniowy na przedmiot Programowanie Zaawansowane
}

\author{Mikołaj Ewald}
\date{2024/2025}

\begin{document}
   
   \maketitle
   
   \noindent\makebox[\linewidth]{\rule{\columnwidth}{0.4pt}}
   
   \section*{Krótki opis aplikacji}
    Aplikacja \textbf{expenses-tracker} ma na celu rozwiązanie problemu śledzenia wydatków. W tym celu każdy użytkownik
    posiadający konto w aplikacji uzyskuje możliwość utworzenia list/y. W stworzonej liście program umożliwia dodanie
    \textit{wydatku} zawierającego m.in. tytuł, datę oraz kwotę. Do każdego wydatku istnieje możliwość dodania zdjęcia
    przedstawiającego otrzymany przy płatności rachunek. \\
    Domyślenie każda lista jest prywatna, jednak może zostać udostępniona innym użytkownikom posiadającym konto w 
    aplikacji. Wynikiem takiej operacji jest uzyskanie możliwoście dodawania lub edycji elementów listy przez wskazanych
    użytkowników. Dodatkową funkcjonalnością jest możliwość zmiany widoczności listy na publiczną, co spowoduje 
    możliwość wyświetlenia elementów listy przez wszystkie osoby posiadające do niej link, również te nieposiadające 
    konta w serwisie.
    
\end{document}